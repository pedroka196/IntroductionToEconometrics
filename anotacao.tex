\documentclass[]{article}
\usepackage{lmodern}
\usepackage{amssymb,amsmath}
\usepackage{ifxetex,ifluatex}
\usepackage{fixltx2e} % provides \textsubscript
\ifnum 0\ifxetex 1\fi\ifluatex 1\fi=0 % if pdftex
  \usepackage[T1]{fontenc}
  \usepackage[utf8]{inputenc}
\else % if luatex or xelatex
  \ifxetex
    \usepackage{mathspec}
  \else
    \usepackage{fontspec}
  \fi
  \defaultfontfeatures{Ligatures=TeX,Scale=MatchLowercase}
\fi
% use upquote if available, for straight quotes in verbatim environments
\IfFileExists{upquote.sty}{\usepackage{upquote}}{}
% use microtype if available
\IfFileExists{microtype.sty}{%
\usepackage{microtype}
\UseMicrotypeSet[protrusion]{basicmath} % disable protrusion for tt fonts
}{}
\usepackage[margin=1in]{geometry}
\usepackage{hyperref}
\hypersetup{unicode=true,
            pdftitle={Anotacao - Capítulo 14},
            pdfauthor={Pedro Henrique Oliveira de Souza},
            pdfborder={0 0 0},
            breaklinks=true}
\urlstyle{same}  % don't use monospace font for urls
\usepackage{color}
\usepackage{fancyvrb}
\newcommand{\VerbBar}{|}
\newcommand{\VERB}{\Verb[commandchars=\\\{\}]}
\DefineVerbatimEnvironment{Highlighting}{Verbatim}{commandchars=\\\{\}}
% Add ',fontsize=\small' for more characters per line
\usepackage{framed}
\definecolor{shadecolor}{RGB}{248,248,248}
\newenvironment{Shaded}{\begin{snugshade}}{\end{snugshade}}
\newcommand{\KeywordTok}[1]{\textcolor[rgb]{0.13,0.29,0.53}{\textbf{#1}}}
\newcommand{\DataTypeTok}[1]{\textcolor[rgb]{0.13,0.29,0.53}{#1}}
\newcommand{\DecValTok}[1]{\textcolor[rgb]{0.00,0.00,0.81}{#1}}
\newcommand{\BaseNTok}[1]{\textcolor[rgb]{0.00,0.00,0.81}{#1}}
\newcommand{\FloatTok}[1]{\textcolor[rgb]{0.00,0.00,0.81}{#1}}
\newcommand{\ConstantTok}[1]{\textcolor[rgb]{0.00,0.00,0.00}{#1}}
\newcommand{\CharTok}[1]{\textcolor[rgb]{0.31,0.60,0.02}{#1}}
\newcommand{\SpecialCharTok}[1]{\textcolor[rgb]{0.00,0.00,0.00}{#1}}
\newcommand{\StringTok}[1]{\textcolor[rgb]{0.31,0.60,0.02}{#1}}
\newcommand{\VerbatimStringTok}[1]{\textcolor[rgb]{0.31,0.60,0.02}{#1}}
\newcommand{\SpecialStringTok}[1]{\textcolor[rgb]{0.31,0.60,0.02}{#1}}
\newcommand{\ImportTok}[1]{#1}
\newcommand{\CommentTok}[1]{\textcolor[rgb]{0.56,0.35,0.01}{\textit{#1}}}
\newcommand{\DocumentationTok}[1]{\textcolor[rgb]{0.56,0.35,0.01}{\textbf{\textit{#1}}}}
\newcommand{\AnnotationTok}[1]{\textcolor[rgb]{0.56,0.35,0.01}{\textbf{\textit{#1}}}}
\newcommand{\CommentVarTok}[1]{\textcolor[rgb]{0.56,0.35,0.01}{\textbf{\textit{#1}}}}
\newcommand{\OtherTok}[1]{\textcolor[rgb]{0.56,0.35,0.01}{#1}}
\newcommand{\FunctionTok}[1]{\textcolor[rgb]{0.00,0.00,0.00}{#1}}
\newcommand{\VariableTok}[1]{\textcolor[rgb]{0.00,0.00,0.00}{#1}}
\newcommand{\ControlFlowTok}[1]{\textcolor[rgb]{0.13,0.29,0.53}{\textbf{#1}}}
\newcommand{\OperatorTok}[1]{\textcolor[rgb]{0.81,0.36,0.00}{\textbf{#1}}}
\newcommand{\BuiltInTok}[1]{#1}
\newcommand{\ExtensionTok}[1]{#1}
\newcommand{\PreprocessorTok}[1]{\textcolor[rgb]{0.56,0.35,0.01}{\textit{#1}}}
\newcommand{\AttributeTok}[1]{\textcolor[rgb]{0.77,0.63,0.00}{#1}}
\newcommand{\RegionMarkerTok}[1]{#1}
\newcommand{\InformationTok}[1]{\textcolor[rgb]{0.56,0.35,0.01}{\textbf{\textit{#1}}}}
\newcommand{\WarningTok}[1]{\textcolor[rgb]{0.56,0.35,0.01}{\textbf{\textit{#1}}}}
\newcommand{\AlertTok}[1]{\textcolor[rgb]{0.94,0.16,0.16}{#1}}
\newcommand{\ErrorTok}[1]{\textcolor[rgb]{0.64,0.00,0.00}{\textbf{#1}}}
\newcommand{\NormalTok}[1]{#1}
\usepackage{graphicx,grffile}
\makeatletter
\def\maxwidth{\ifdim\Gin@nat@width>\linewidth\linewidth\else\Gin@nat@width\fi}
\def\maxheight{\ifdim\Gin@nat@height>\textheight\textheight\else\Gin@nat@height\fi}
\makeatother
% Scale images if necessary, so that they will not overflow the page
% margins by default, and it is still possible to overwrite the defaults
% using explicit options in \includegraphics[width, height, ...]{}
\setkeys{Gin}{width=\maxwidth,height=\maxheight,keepaspectratio}
\IfFileExists{parskip.sty}{%
\usepackage{parskip}
}{% else
\setlength{\parindent}{0pt}
\setlength{\parskip}{6pt plus 2pt minus 1pt}
}
\setlength{\emergencystretch}{3em}  % prevent overfull lines
\providecommand{\tightlist}{%
  \setlength{\itemsep}{0pt}\setlength{\parskip}{0pt}}
\setcounter{secnumdepth}{0}
% Redefines (sub)paragraphs to behave more like sections
\ifx\paragraph\undefined\else
\let\oldparagraph\paragraph
\renewcommand{\paragraph}[1]{\oldparagraph{#1}\mbox{}}
\fi
\ifx\subparagraph\undefined\else
\let\oldsubparagraph\subparagraph
\renewcommand{\subparagraph}[1]{\oldsubparagraph{#1}\mbox{}}
\fi

%%% Use protect on footnotes to avoid problems with footnotes in titles
\let\rmarkdownfootnote\footnote%
\def\footnote{\protect\rmarkdownfootnote}

%%% Change title format to be more compact
\usepackage{titling}

% Create subtitle command for use in maketitle
\newcommand{\subtitle}[1]{
  \posttitle{
    \begin{center}\large#1\end{center}
    }
}

\setlength{\droptitle}{-2em}

  \title{Anotacao - Capítulo 14}
    \pretitle{\vspace{\droptitle}\centering\huge}
  \posttitle{\par}
    \author{Pedro Henrique Oliveira de Souza}
    \preauthor{\centering\large\emph}
  \postauthor{\par}
      \predate{\centering\large\emph}
  \postdate{\par}
    \date{7 de março de 2019}


\begin{document}
\maketitle

\subsection{Capítulo 14}\label{capitulo-14}

O capítulo introduz dados de série temporal e correlação temporal.
Começa apresentando gráficos com a inflação dos EUA.

A observação da variável \(Y\) no momento \(t\) se mostra como \(Y_t\).
O total de observações é \(T\). O espaço entre \(t\) e \(t+1\) é um
intervalo abstrato qualquer.

\begin{itemize}
\tightlist
\item
  A variação entre os períodos \(Y_t\) e \(Y_{t-1}\) é chamada de
  \textbf{primeira diferença}. Ou seja:
\end{itemize}

\[\Delta Y_t=Y_t - Y_{t-1}\]

É recomendado tomar o \(log\) das séries temporais. Algumas séries
apresentam crescimento aproxidamente exponencial, sendo recomendado por
isso. Além disso, desvios padrões de mudas séries temporais econômicas é
aproximadamente proporcional ao seu nível, isto é, o desvio padrão pode
ser expresso corretamente uma porcentagem do nível das séries. Então o
desvio padrão do logaritmo das séries é aproximadamente constante.

Este começo usados do \emph{dataset} \textbf{USMacroSW}.

\begin{Shaded}
\begin{Highlighting}[]
\CommentTok{# load US macroeconomic data}
\NormalTok{USMacroSWQ <-}\StringTok{ }\KeywordTok{read_xlsx}\NormalTok{(}\StringTok{"Data_StockWatson/us_macro_quarterly.xlsx"}\NormalTok{,}
                        \DataTypeTok{sheet =} \DecValTok{1}\NormalTok{,}
                        \DataTypeTok{col_types =} \KeywordTok{c}\NormalTok{(}\StringTok{"text"}\NormalTok{, }\KeywordTok{rep}\NormalTok{(}\StringTok{"numeric"}\NormalTok{, }\DecValTok{9}\NormalTok{)))}

\KeywordTok{colnames}\NormalTok{(USMacroSWQ) <-}\StringTok{ }\KeywordTok{c}\NormalTok{(}\StringTok{"Date"}\NormalTok{, }\StringTok{"GDPC96"}\NormalTok{, }\StringTok{"JAPAN_IP"}\NormalTok{, }\StringTok{"PCECTPI"}\NormalTok{, }
                          \StringTok{"GS10"}\NormalTok{, }\StringTok{"GS1"}\NormalTok{, }\StringTok{"TB3MS"}\NormalTok{, }\StringTok{"UNRATE"}\NormalTok{, }\StringTok{"EXUSUK"}\NormalTok{, }\StringTok{"CPIAUCSL"}\NormalTok{)}

\CommentTok{# format date column}
\NormalTok{USMacroSWQ}\OperatorTok{$}\NormalTok{Date <-}\StringTok{ }\KeywordTok{as.yearqtr}\NormalTok{(USMacroSWQ}\OperatorTok{$}\NormalTok{Date, }\DataTypeTok{format =} \StringTok{"%Y:0%q"}\NormalTok{)}
\end{Highlighting}
\end{Shaded}

Logo no começo, é usado o pacote \textbf{xts}, que cria um objeto de
série de tempo com dados crus. Seu uso é com:

\begin{itemize}
\tightlist
\item
  \textbf{xts}(\textbf{x}= \emph{dados}, \textbf{order.by} = \emph{vetor
  único de datas/tempos})
\end{itemize}

\begin{Shaded}
\begin{Highlighting}[]
\CommentTok{# GDP series as xts object}
\NormalTok{GDP <-}\StringTok{ }\KeywordTok{xts}\NormalTok{(USMacroSWQ}\OperatorTok{$}\NormalTok{GDPC96, USMacroSWQ}\OperatorTok{$}\NormalTok{Date)[}\StringTok{"1960::2013"}\NormalTok{]}

\CommentTok{# GDP growth series as xts object}
\NormalTok{GDPGrowth <-}\StringTok{ }\KeywordTok{xts}\NormalTok{(}\DecValTok{400} \OperatorTok{*}\StringTok{ }\KeywordTok{log}\NormalTok{(GDP}\OperatorTok{/}\KeywordTok{lag}\NormalTok{(GDP)))}
\end{Highlighting}
\end{Shaded}


\end{document}
